
\documentclass[letterpaper, 11pt]{article}
\usepackage[utf8]{inputenc}
\usepackage{titlesec}
\usepackage{fullpage} % changes the margin
\usepackage{graphicx} %package to manage images
\graphicspath{ {./images/} }

\begin{document}

\begin{titlepage}
\vspace*{0.7in}
\begin{center}
\begin{figure}[htb]
\begin{center}
\includegraphics[width=8cm]{univ_logo}
\end{center}
\end{figure}
\vspace*{0.3in}
\begin{Large}
\textbf{SOEN 6011 : SOFTWARE ENGINEERING PROCESSES} \\
\end{Large}
\vspace*{0.1in}
\begin{Large}
\textbf{SUMMER 2022} \\
\end{Large}
\vspace*{0.9in}
\begin{Large}
\textbf{Eternity} \\
\end{Large}
\vspace*{0.625in}
\begin{Large} 


\textbf{PROBLEM - 2} \\
\vspace*{0.2in}
Requirements
\vspace*{0.1in}
\\\footnotesize{ISO/IEC/IEEE} 29148 Standard \\
\end{Large}
\vspace*{0.625in}
\rule{80mm}{0.1mm}\\
\vspace*{0.1in}
\begin{large}
Author \\
\vspace*{0.1in}
Neona Sheetal Pinto\\
\vspace*{1.0in}
\date{\normalsize\today} 
\end{large}
\end{center}
\begin{center}
https://www.overleaf.com/project/62e938922b3937446fb7f547\end{center}
\end{titlepage}

\tableofcontents

\newpage
\section{{PROBLEM 2 - F6: $B(x,y)$}}
    \subsection{Assumption:} 
        To compute the value of the Beta function, we can estimate the value of the definite integral using numerical methods.
    \subsection{Requirements:}
        The current section describes the requirements to implement the function $B(x, y)$.

        \subsubsection{\textbf{Requirement Id : R1}}
            \begin{tabular}{ll}
                \textbf{Id} & R1 \\
                \textbf{Overview} & $B(x,y)$, where x, y $\in R, x, y > 0.$ \\
                \textbf{Version} & 1.0 \\
                \textbf{Description} & 
                \begin{tabular}[c]{@{}l@{}} The Beta function can give output only if both x, y values \\
                entered by user are positive real numbers.
            \end{tabular} \\
                \textbf{Priority} & High \\
                \textbf{Type} & Functional \\
                \textbf{Difficulty} & High \\
                \textbf{Verification} & UserInput\_numericInputCheckTest\_1\\
            \end{tabular}
            
        \subsubsection{\textbf{Requirement Id : R2}}
            \begin{tabular}{ll}
                \textbf{Id} & R2 \\
                \textbf{Overview} & ${\displaystyle \mathrm {B} (x,y)=\int _{0}^{1}t^{x-1}(1-t)^{y-1}\,dt}$\\
                \textbf{Version} & 1.0 \\
                \textbf{Description} & 
                \begin{tabular}[c]{@{}l@{}} To compute the value of Beta function for any real number
                \\we need to be able to compute definite integral as 
                \\defined in the mathematical realm of calculus.
            \end{tabular} \\
                \textbf{Priority} & High \\
                \textbf{Type} & Functional \\
                \textbf{Difficulty level} & Medium \\
                \textbf{Verification} & Integral\_testBetaFunctionWithDoubleValues\_2\\
            \end{tabular}

        \subsubsection{\textbf{Requirement Id : R3}}
            \begin{tabular}{ll}
                \textbf{Id} & R3 \\
                \textbf{Overview} & To calculate Beta function, exponent function $A^B$ is required.\\
                \textbf{Version} & 1.0 \\
                \textbf{Description} & 
                \begin{tabular}[c]{@{}l@{}} To calculate the Beta function, a subordinate function 
                \\needs to be used to calculate the value A raised to the power B.
            \end{tabular} \\
                \textbf{Priority} & High \\
                \textbf{Type} & Functional \\
                \textbf{Difficulty level} & Medium \\
                \textbf{Verification} &Exponent\_positiveNumberPowerofPositiveNumber\_3\\
            \end{tabular}

        \subsubsection{\textbf{Requirement Id : R4}}
            \begin{tabular}{ll}
                \textbf{Id} & R4 \\
                \textbf{Overview} & Accuracy\\
                \textbf{Version} & 1.0 \\
                \textbf{Description} & 
                \begin{tabular}[c]{@{}l@{}} To accurately compute the value of Beta function,
                \\for large inputs of x and y, we need to have the ability to
                \\store large numbers.
            \end{tabular} \\
                \textbf{Priority} & High \\
                \textbf{Type} & Non-Functional \\
                \textbf{Difficulty level} & Medium \\
            \end{tabular}
        \subsubsection{\textbf{Requirement Id : R5}}
            \begin{tabular}{ll}
                 \textbf{Id} & R5 \\
                \textbf{Overview} & Scalable \\ 
                \textbf{Version} & 1.0 \\
                \textbf{Description} & 
                \begin{tabular}[c]{@{}l@{}} The method used to calculate the Beta function, should be 
                \\scalable for different input values and hardware requirements.\\
            \end{tabular} \\
                \textbf{Priority} & High \\
                \textbf{Type} & Functional \\
                \textbf{Difficulty level} & Medium \\
            \end{tabular}

         \subsubsection{\textbf{Requirement Id : R6}}
            \begin{tabular}{ll}
             \textbf{Id} & R6 \\
            \textbf{Overview} & Performance  \\
            \textbf{Version} & 1.0 \\
            \textbf{Description} & 
            \begin{tabular}[c]{@{}l@{}} The method used for the Beta function, should be optimized 
            \\for performance so that if efficiently calculates the integral 
            \\for large inputs values. \\
            \end{tabular} \\
            \textbf{Priority} & High \\
            \textbf{Type} & Functional \\
            \textbf{Difficulty level} & Medium \\
            \end{tabular}
                 
        \subsubsection{\textbf{Requirement Id : R7}}
            \begin{tabular}{ll}
             \textbf{Id} & R7 \\
            \textbf{Overview} & Usability  \\
            \textbf{Version} & 1.0 \\
            \textbf{Description} & 
            \begin{tabular}[c]{@{}l@{}} The command Line Interface is user friendly,\\ guiding the user step by step how to move about. Error messages are understandable.\\ The application is easy to use and friendly.
            \end{tabular} \\
            \textbf{Priority} & High \\
            \textbf{Type} & Functional \\
            \textbf{Difficulty level} & Easy \\
            \end{tabular}
\\
\section{Annexure:}
    \begin{itemize}
      \item \textbf{Trello Board :} \textit{https://trello.com/eternity119}
      \item \textbf{Code Version Control :} \textit{https://github.com/neonapinto/Scientific\_calculator}
      \item \textbf{Overleaf :} \textit{https://www.overleaf.com/project/62e938922b3937446fb7f547}
    \end{itemize}
        
\begin{thebibliography}{}
\bibitem{ReqView} 
ReqView : Nykamp DQ: Requirements Specification Templates
\\\texttt{https://www.reqview.com/doc/iso-iec-ieee-29148-templates}
\bibitem{29148} 
29148-2018-ISO/IEC/IEEE International Standard-Systems and software engineering-Life cycle processes-Requirements engineering,
\\\texttt{https://standards.ieee.org/standard/29148-2018.html}
\end{thebibliography}
\end{document}